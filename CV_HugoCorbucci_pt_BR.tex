\documentclass[letter,10pt]{article}

%A Few Useful Packages
\usepackage{marvosym}
\usepackage{fontspec} 					%for loading fonts
\usepackage{xunicode,xltxtra,url,parskip} 	%other packages for formatting
\RequirePackage{color,graphicx}
\usepackage[usenames,dvipsnames]{xcolor}
\usepackage{fullpage}
\usepackage{supertabular} 				%for Grades
\usepackage{titlesec} % custom \section

%Setup hyperref package, and colours for links
\usepackage{hyperref}
\definecolor{linkcolour}{rgb}{0,0.2,0.6}
\hypersetup{colorlinks,breaklinks,urlcolor=linkcolour, linkcolor=linkcolour}

%FONTS
\defaultfontfeatures{Mapping=tex-text}
%\setmainfont[SmallCapsFont = Fontin SmallCaps]{Fontin}

%CV Sections inspired by: 
%http://stefano.italians.nl/archives/26
\titleformat{\section}{\Large\scshape\raggedright}{}{0em}{}[\titlerule]
\titlespacing{\section}{0pt}{3pt}{3pt}
%Tweak a bit the top margin
%\addtolength{\voffset}{-1.3cm}

%--------------------BEGIN DOCUMENT----------------------
\begin{document}

%--------------------TITLE-------------
\par{\centering
		{\Huge Hugo \textsc{Corbucci}
	}\bigskip\par}

%--------------------SECTIONS-----------------------------------
%Section: Personal Data
\section{Informações Pessoais}

\begin{tabular}{p{2.5cm}l}
    \textsc{Nacionalidade:} & Franco-Brasileiro
    \\
    \textsc{Nascimento:} & 26 de Dezembro de 1983\\
    \textsc{Mora em:}   & São Paulo, Brasil \\
    \textsc{Email:}     &
    \href{mailto:hugo.corbucci@gmail.com}{hugo.corbucci@gmail.com}\\
    \textsc{Blog:}     &
    \href{http://codeache.blogspot.com}{http://codeache.blogspot.com}
\end{tabular}

% Section: Education
\section{Educação}
\begin{tabular}{p{2.5cm}l}
  \emph{Atual} & Mestrado em \textsc{Ciências da Computação}\\
  \textsc{2007} & \textbf{Universidade do Estado de São Paulo}, São Paulo\\
  & Dissertação: ``Métodos ágeis e software livre: Um estudo do relacionamento entre estas duas comunidades''\\
  & \small Orientador: Prof. Alfredo \textsc{Goldman}\\
\end{tabular}

\begin{tabular}{p{2.5cm}l}
  \textsc{2006} & Graduação em \textsc{Ciências da Computação}\\
  \textsc{2003} &\normalsize\textbf{Universidade do Estado de São Paulo}, São Paulo\\
  & Dissertação: ``Archimedes: An Open Source CAD developed with\\
  & eXtreme Programming and Object Orientation''\\
  & \small Orientador: Fabio \textsc{Kon}\\
  & ``Aluno Destaque'' pela Sociedade Brasileira de Computação\\
\end{tabular}

\begin{tabular}{p{2.5cm}l}
  \textsc{2001} & \textbf{Lycée Pasteur}, São Paulo\\
  \textsc{1990} & Colégio Experimental bilíngue Português-Francês\\
  & Exame de aprovação Francês \textit{Baccalaureat Scientifique}: 14/20
\end{tabular}

\begin{tabular}{p{2.5cm}l}
  \textsc{1990} & \textbf{Escola Pública Primária Foyatier}, Paris\\
  \textsc{1986} & Escola primária no ensino Francês\\
\end{tabular}

% Section: Languages
\section{Línguas}
\begin{tabular}{p{2.5cm}l}
 \textsc{Português:}&Nativo\\
 \textsc{Francês:}&Nativo\\
 \textsc{Inglês:}&Fluente\\
 \textsc{Espanhol:}&Conhecimentos básicos\\
\end{tabular}

\section{Conhecimentos com Linguagens de Programação}
\begin{tabular}{p{2.5cm}l}
 Básico:& C, C++, Python, C\#, Haskell\\
 Intermediário:& Smalltalk, Javascript\\
 Avançado:& Java, Ruby\\
\end{tabular}

\section{Interesses e Tópicos de Pesquisa}

\begin{itemize}
\item Métodos Ágeis
\item ``Ecossistema'' do Software Livre
\item \textit{Dojo}s de Programação
\item Ambientes Integrados de Desenvolvimento (\textit{IDE})
\item Código Limpo
\item Testes automatizados de usuários na interface gráfica
\item Sistemas Multi-linguagens
\item Sistemas tolerantes a falhas
\item Linguagens de Programação com sintaxes mais próximas das
  linguagens naturais
\end{itemize}

\section{Destaques}

\begin{itemize}
\item Programador generalista com interesse tanto nos aspectos
  técnicos quanto sociais da programação.
\item Desenvolvedor de software em diversas aplicações (de aplicações
  \textit{web} até \textit{desktop}, linha de comando até interfaces
  gráficas passando por serviços).
\item Altamente qualificado em programação orientada a objetos
  (encapsulamento, coesão, padrões de projeto, UML, testes de unidade,
  testes de integração, \textit{mocks/stubs} e convenções).
\item Altamente qualificado em tecnologias Java (SWT, \textit{Eclipse
    Rich Client Platform}, Ant, JUnit, Swing, VRaptor).
\item Altamente qualificado em tecnologias \textit{Web} (\textit{Ruby
    on Rails}, VRaptor, GWT, AJAX, JSON, Javascript, CSS, HTML, haml,
  sass, x-path, jquery, prototype).
\item Experiente em programação em par e no papel de \textit{coach}
  ágil.
\item Acostumado tanto a ambientes de trabalho formais e muito
  informais.
\end{itemize}

% Section: Work Experience
\section{Experiência de Trabalho}

\begin{tabular}{p{2.5cm}|p{13.5cm}}
  \emph{Atual} & Fundador da \textsc{Agilbits}\\
  \textsc{Mar 2008}& \emph{Desenvolvedor e Consultor Ágil}\\
  &\\
  &A Agilbits foi criada para suprir as necessidades do mercado
  Brasileiro em termos de treinamento, \textit{coaching} e
  desenvolvimento de software de alta qualidade. Atuei como o
  administrador da empresa assim como desenvolvedor, consultor e
  \textit{coach}.
  Dentre os diversos projetos desenvolvidos, tivemos diversas
  experiências com \textit{Ruby on Rails} desde sistemas de gestão de
  conteúdo (\textit{CMS}) simples até processamento de imagens e
  sistemas de submissões.
  Do lado do \textit{Java}, atuei como membro de uma equipe de 3 a 5
  pessoas que liderou desde a fase de exploração de um projeto até o
  desenvolvimento completo de um produto em mais de 25 iterações. O
  projeto é uma aplicação \textit{desktop} baseada na plataforma do
  Eclipse que atingiu mais de 40000 linhas de código de produção e
  45000 linhas de testes automatizados.
\end{tabular}

\begin{tabular}{p{2.5cm}|p{13.5cm}}
  \emph{Atual} & Consultor na \textsc{AgilCoop}\\
  \textsc{Jan 2007}& \emph{Instrutor e Coach}\\
  &\\
  &A AgilCoop foi fundada por professores, estudantes e ex-estudantes
  da Universidade do Estado de São Paulo (USP) para promover e
  divulgar os valores e princípios do desenvolvimento ágil de software
  no Brasil. Atuei como instrutor durante cursos de verão em cursos
  como ``Introdução a Métodos Ágeis de Desenvolvimento de Software'',
  ``Desenvolvimento de Software de Qualidade através de Testes
  Automatizados'' e ``Laboratório de Programação eXtrema'' em São
  Paulo, São Carlos e Salvador.
  Também trabalhei como \textit{coach} em diversos projetos na
  academia e na indústria. A AgilCoop também é responsável pela
  organização do ``Encontro Ágil'' do qual foi co-organizador em 2008,
  2009 e 2010. Esse evento tem como público alvo a comunidade local de
  pessoas interessadas em métodos ágeis que procuram por contato com
  pessoas mais experientes.
\end{tabular}

\begin{tabular}{p{2.5cm}|p{13.5cm}}
  \emph{Atual} & Organizador da \textsc{Agile Brazil}\\
  \textsc{Set 2009}& \emph{Webmaster, desenvolvedor e membro da equipe
    de avaliação}\\
  &\\
  &Na Agile 2009, aproximadamente 8 Brasileiros se juntaram e
  decidiram organizar uma conferência nacional sobre métodos ágeis no
  Brasil. Atuei como membro do comitê organizador desde o início
  ajudando a formar o comitê de 15 pessoas, a planejar e implementar a
  estratégia de marketing, a entrar em contato com patrocinadores e a
  elaborar o programa da conferência. Em Junho de 2010, a conferência
  recebeu mais de 800 pessoas que submeteram mais de 150 propostas das
  quais em torno de 40 foram selecionadas. Trabalhei com o Danilo
  \textsc{Sato} no desenvolvimento do sistema de submissões em
  \textit{Ruby on Rails}.
\end{tabular}

\begin{tabular}{p{2.5cm}|p{13.5cm}}
  \textsc{Mar - Jun} & \textsc{Instituto de Matemática e Estatística
    (IME/USP)} - Monitor\\
  \textsc{2008 a 2010}& \emph{Coach de XP e programador}\\
  &\\
  & Fui monitor da disciplina de Programação Extrema (XP) por 3 anos
  seguidos. Atuei como \textit{Coach} e ajudei mais de 5 equipes em
  cada ano. Ajudei a implantar um scrum de scrums para permitir o
  aumento de inscritos na disciplina de 40 estudantes para mais de 60
  e de 5 equipes para mais de 8.
\end{tabular}

\begin{tabular}{p{2.5cm}|p{13.5cm}}
  \textsc{Mar - Jun} & \textsc{Instituto de Matemática e Estatística
    (IME/USP)} - Monitor\\
  \textsc{2007 a 2009}& \emph{Palestrante, revisor de código e crítico
    de código}\\
  &\\
  &Monitor das disciplinas de Programação Orientada a Objetos e
  Programação Concorrente por 3 anos seguidos. Revisei e corrigi
  códigos de alunos de graduação e de pós-graduação assim como dei
  algumas aulas sob cada um dos assuntos.
\end{tabular}


\begin{tabular}{p{2.5cm}|p{13.5cm}}
  \textsc{Set 2007} & Desenvolvedor na \textsc{MAPS Risk
    Management Solutions} \\
  \textsc{Dez 2006} &\emph{Desenvolvedor de Software and Coach Ágil}\\
  &\\
  & A MAPS é uma empresa que provê soluções para análise de risco do
  mercado financeiro. Eu me juntei à equipe que estava desenvolvendo o
  novo sistema destinado a substituir o antigo. Trabalhei numa equipe
  de 8 pessoas como desenvolvedor e \textit{coach} ágil para montar
  uma aplicação \textit{web} em Java usando \textit{Java Server
    Faces}. Ajudei a implantar um pouco de programação em par, testes
  de unidades e quadros de acompanhamento.
\end{tabular}

\begin{tabular}{p{2.5cm}|p{13.5cm}}
  \textsc{Dez 2006} & \textsc{Banco Votorantim} - Estágio\\
  \textsc{Dez 2005} &\emph{Desenvolvedor de Software}\\
  &\\
  &Entrei nessa equipe de 8 pessoas como o responsável de TI sob a
  supervisão de 2 analistas de negócio e 6 especialistas da área. Apesar de
  ter sido contratado para manter a aplicação existente, quando a
  maioria dos membros da equipe saiu da empresa, fiquei responsável
  por ajudar os analistas de negócio a manter a área funcionando.
\end{tabular}

\begin{tabular}{p{2.5cm}|p{13.5cm}}
  \textsc{Mar - Jun} & \textsc{Instituto de Matemática e Estatística
    (IME/USP)} - \textit{Coach} de XP\\
  \textsc{2006 e 2007}& \emph{Líder de equipe e programador}\\
  &\\
  &  Trabalhei como \textit{Coach} de XP, líder de equipe e
  programador na disciplina de Programação Extrema (XP) por 2 anos
  seguidos. Ajudei duas equipes bem diferentes a desenvolver o
  \emph{Archimedes - O CAD Aberto}  do nada até uma aplicação baseada
  no SWT em Java e, então, evolui-la para uma aplicação baseada na
  plataforma do Eclipse. Coordenei uma equipe de 8 e depois de 7
  estudantes de graduação em 2006 e 2007 respectivamente.
\end{tabular}

\begin{tabular}{p{2.5cm}|p{13.5cm}}
  \textsc{Out 2005} & \textsc{Instituto de Matemática e Estatística
    (IME/USP)} - Iniciação Científica\\
  \textsc{Jan 2005}& \emph{Desenvolvedor de Ontologia}\\
  &\\
  & A Prof. Dra. Renata \textsc{Wasserman} e o Prof. Dr. Fabio
  \textsc{Kon} atuaram como meus orientadores para elaboração de uma
  ontologia sobre orientação a objetos. O trabalho serviu como base
  para uma busca semântica textual em alguns vídeos de aulas dadas por
  Joe Yoder.
\end{tabular}

\begin{tabular}{p{2.5cm}|p{13.5cm}}
  \textsc{Dez 2003} & \textsc{Cidade do Conhecimento} - Estágio\\
  \textsc{Jun 2003}& \emph{Webmaster}\\
  &\\
  & Trabalhei como \textit{webmaster} na Cidade do Conhecimento
  mantendo um sistema de ensino a distância desenvolvido em PHP.
\end{tabular}

\section{Publicações}

\begin{itemize}
\item \textbf{Open Source and Agile Methods: Two worlds closer than it
    seems.} Hugo Corbucci, Alfredo Goldman.

  \textit{Anais da 11a Conferência Internacional sobre Desenvolvimento
    Ágil de Software (XP 2010), Trondheim, Noruega.} Junho 2010.

\item \textbf{Prototypes Are Forever: Evolving from a Prototype
    Project to a Full-Featured System} Hugo Corbucci, Mariana Vivian
  Bravo, Alexandre Freire da Silva, Fernando Freire da Silva.

  \textit{Anais da 11a Conferência Internacional sobre Desenvolvimento
    Ágil de Software (XP 2010), Trondheim, Noruega.} Junho 2010.

\item \textbf{Open Source and Agile: Two worlds that should have a
    closer interaction} Hugo Corbucci, Alfredo Goldman.

  \textit{Anais do V Workshop Latino-Americano sobre Engenharia de
    Software Experimental, 2008 (ESELAW 2008), Salvador, Brasil}
  Novembro 2008.

\item \textbf{Coding Dojo: an environment for learning and sharing
    Agile practices} Danilo Sato, Hugo Corbucci, Mariana Bravo.

  \textit{Anais da Conferência Ágil 2008 (Agile 2008), Toronto,
    Canadá.} Agosto 2008.

\item \textbf{Archimedes ­ o CAD Aberto: Uma aplicação para desenho
    técnico baseada na plataforma do Eclipse} Hugo Corbucci, Mariana
  Vivian Bravo.

  \textit{Anais do VII Workshop de Software Livre, 2007 (WSL 2007),
    Porto Alegre, Brasil.} Abril 2007.

\item \textbf{Archimedes: Um CAD Livre desenvolvido com programação
    extrema e orientação a objetos} Hugo Corbucci, Mariana Vivian
  Bravo. Orientador: Fabio Kon.

  \textit{Trabalho de conclusão de curso de Bacharelado em Ciências da
    Computação no IME/USP, São Paulo, Brasil.} Dezembro 2006.
\end{itemize}

\section{Atividades}

\begin{tabular}{p{2.5cm}l}
  \textsc{Ago 2010} & Palestrante no \textbf{TDC 2010}\\
  &Dando uma palestra sobre ``Código Limpo'' em São Paulo, SP, Brasil.\\
\end{tabular}

\begin{tabular}{p{2.5cm}l}
  \textsc{Ago 2010} & Palestrante no \textbf{DevInSampa 2010}\\
  &Dando uma palestra sobre ``Código Limpo'' em São Paulo, SP, Brasil.\\
\end{tabular}

\begin{tabular}{p{2.5cm}l}
  \textsc{Jun 2010} & Palestrante na \textbf{Agile Brazil 2010}\\
  &Dando um tutorial sobre ``Retrospectivas Ágeis'' em Porto Alegre, RS, Brasil.\\
\end{tabular}

\begin{tabular}{p{2.5cm}l}
  \textsc{Jun 2010} & Palestrante no \textbf{XP 2010}\\
  &Dando um \textit{workshop} de 180 minutos sobre ``Dojo de
  Programação - Formato Kake'',\\
  & apresentando um relato de experiência ``Protótipos são para sempre''\\
  & com Alexandre \textsc{Freire} e Mariana \textsc{Bravo},\\
  & e um Kata com Danilo \textsc{Sato} em Smalltalk em Trondheim, Noruega.\\
\end{tabular}

\begin{tabular}{p{2.5cm}l}
  \textsc{Dez 2009} & Palestrante no \textbf{1o Encontro de Software
    Livre da Dataprev}\\
  &Dando uma palestra sobre ``Métodos Ágeis no contexto do Software Livre''\\
  &em Campo Grande, MS, Brasil.\\
\end{tabular}

\begin{tabular}{p{2.5cm}l}
  \textsc{Dez 2009} & Palestrante no \textbf{4o SOLISC}\\
  &Dando uma palestra sobre ``Archimedes - o CAD Aberto'' em Florianópolis, SC, Brasil.\\
\end{tabular}

\begin{tabular}{p{2.5cm}l}
  \textsc{Nov 2009} & Palestrante no \textbf{ESELAW 2008}\\
  &Dando um curso sobre ``Métodos Ágeis'' e\\
  &uma palestra sobre ``Métodos Ágeis no contexto do Software Livre'' em Salvador, BA, Brasil.\\
\end{tabular}

\begin{tabular}{p{2.5cm}l}
  \textsc{Out 2009} & Palestrante no \textbf{Encontro Ágil 2009}\\
  &Dando uma palestra sobre ``Métodos Ágeis no contexto do Software Livre'' e\\
  &um \textit{workshop} sobre ``Jogo de Lego sobre Lean'' do trabalho de\\
  &Francisco \textsc{Trindade} e Danilo \textsc{Sato} em São Paulo, SP, Brasil.\\
\end{tabular}

\begin{tabular}{p{2.5cm}l}
  \textsc{Ago 2009} & Voluntário na \textbf{Agile 2009}\\
  & com bolsa acadêmica em Chicago, IL, EUA.\\
\end{tabular}

\begin{tabular}{p{2.5cm}l}
  \textsc{Jun 2009} & Palestrante no \textbf{FISL 2009}\\
  &Dando uma palestra sobre ``Archimedes - o CAD Aberto'',\\
  & um tutorial sobre ``Eclipse RCP - Aplicações \textit{desktop} nativas'' e\\
  & um \textit{workshop} sobre ``Dojo de Programação''\\
  & no 10o Fórum Internacional de Software Livre em Porto
  Alegre, RS, Brasil.
\end{tabular}

\begin{tabular}{p{2.5cm}l}
  \textsc{Abr 2009} & Avaliador de propostas para o \textbf{FISL 2009}.\\
\end{tabular}

\begin{tabular}{p{2.5cm}l}
  \textsc{Fev 2009} & Palestrante no \textbf{2o Techday da Locaweb}\\
  & Apresentando um Kata sobre ``Rails + RSpec + Cucumber'' em São Paulo, SP, Brasil.\\
\end{tabular}

\begin{tabular}{p{2.5cm}l}
  \textsc{Jan 2009} & Participante da \textbf{Campus Party 2009}\\
  &em São Paulo, SP, Brasil.\\
\end{tabular}

\begin{tabular}{p{2.5cm}l}
  \textsc{Out 2008} & Estudante voluntário na \textbf{OOPSLA'08}\\
  & com bolsa nível Ouro em Nashville, TN, EUA.\\
\end{tabular}

\begin{tabular}{p{2.5cm}l}
  \textsc{Set 2008} & Palestrante na \textbf{PyCon Brasil 2008}\\
  & Dando um \textit{workshop} sobre ``Dojo de programação'' no Rio de Janeiro, RJ, Brasil.\\
\end{tabular}

\begin{tabular}{p{2.5cm}l}
  \textsc{Ago 2008} & Palestrante e Voluntário na \textbf{Agile 2008}\\
  &Apresentando um relato de experiência ``Coding Dojo: An Environment for
  Learning and\\
  & Sharing Agile Practices'' com Danilo \textsc{Sato} e Mariana \textsc{Bravo} e\\
  & trabalhando como voluntário com bolsa acadêmica em Toronto, ON, Canadá.\\
\end{tabular}

\begin{tabular}{p{2.5cm}l}
  \textsc{Jun 2008} & Palestrante no \textbf{FISL 2008}\\
  &Dando um \textit{workshop} sobre ``Dojo de Programação''\\
  & no 9o Fórum Internacional de Software Livre em Porto
  Alegre, RS, Brasil.\\
\end{tabular}

\begin{tabular}{p{2.5cm}l}
  \textsc{Abr 2008} & Avaliador de propostas para o \textbf{FISL 2008}.\\
\end{tabular}

\begin{tabular}{p{2.5cm}l}
  \textsc{Mar 2008} & Palestrante no \textbf{Campus Party 2008}\\
  &Dando um \textit{workshop} sobre ``Archimedes - o CAD Aberto'' em São Paulo, SP, Brasil.\\
\end{tabular}

\begin{tabular}{p{2.5cm}l}
  \textsc{Jan 2008} & Palestrante no \textbf{GrupySP}\\
  &Dando um \textit{workshop} sobre ``Dojo de Programação'' em São Paulo, SP, Brasil.\\
\end{tabular}

\begin{tabular}{p{2.5cm}l}
  \textsc{Nov 2007} & 12o lugar na final Brasileira da
  \textbf{Maratona de Programação da ACM-ICPC}\\
  & com Jeferson \textsc{Rodrigues da Silva} e Marcio
  \textsc{Takashi Iura Oshiro}\\
  & em Belo Horizonte, MG, Brasil.\\
\end{tabular}

\begin{tabular}{p{2.5cm}l}
  \textsc{Out 2007} & Estudante Voluntário na \textbf{OOPSLA'07}\\
  & com bolsa nível Ouro em Montreal, QC, Canadá.\\
\end{tabular}

\begin{tabular}{p{2.5cm}l}
  \textsc{Mai 2007} & Participante do \textbf{FISL 2007}\\
  & o 8o Fórum Internacional de Software Livre em Porto
  Alegre, RS, Brasil.\\
\end{tabular}

\begin{tabular}{p{2.5cm}l}
  \textsc{Jul 2007} & Co-fundador do \textbf{Coding Dojo São Paulo}\\
  & com Danilo \textsc{Sato} e Mariana \textsc{Bravo}.\\
\end{tabular}

\begin{tabular}{p{2.5cm}l}
  \textsc{Abr 2006} & Participante do \textbf{FISL 2006}\\
  & o 7o Fórum Internacional de Software Livre em Porto
  Alegre, RS, Brasil.\\
\end{tabular}

\begin{tabular}{p{2.5cm}l}
  \textsc{Jan 2006} & Fundador do \textbf{Archimedes: o CAD Aberto}\\
  & usando Java e SWT.\\
\end{tabular}

\begin{tabular}{p{2.5cm}l}
  \textsc{Nov 2005} & 16o lugar na final Brasileira da
  \textbf{Maratona de Programação da ACM-ICPC}\\
  & com Jeferson \textsc{Rodrigues da Silva} e Marcio
  \textsc{Takashi Iura Oshiro}\\
  & em Riberão Preto, SP, Brasil.\\
\end{tabular}

\begin{tabular}{p{2.5cm}l}
  \textsc{Mar 2005} & Participante do \textbf{FISL 2005}\\
  & o 6o Fórum Internacional de Software Livre em Porto
  Alegre, RS, Brasil.\\
\end{tabular}

\section{Diversidade}

Ficção Científica, Escalada, Viajar, Teatro e Cinema.

\end{document}
