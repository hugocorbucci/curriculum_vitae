% !TEX program = XeLaTeX
\documentclass[letter,10pt]{article}

%A Few Useful Packages
\usepackage{marvosym}
\usepackage{fontspec} 					%for loading fonts
\usepackage{xunicode,xltxtra,url,parskip} 	%other packages for formatting
\RequirePackage{color,graphicx}
\usepackage[usenames,dvipsnames]{xcolor}
\usepackage{fullpage}
\usepackage{supertabular} 				%for Grades
\usepackage{titlesec} % custom \section

%Setup hyperref package, and colours for links
\usepackage{hyperref}
\definecolor{linkcolour}{rgb}{0,0.2,0.6}
\hypersetup{colorlinks,breaklinks,urlcolor=linkcolour, linkcolor=linkcolour}

%FONTS
\defaultfontfeatures{Mapping=tex-text}
%\setmainfont[SmallCapsFont = Fontin SmallCaps]{Fontin}

%CV Sections inspired by: 
%http://stefano.italians.nl/archives/26
\titleformat{\section}{\Large\scshape\raggedright}{}{0em}{}[\titlerule]
\titlespacing{\section}{0pt}{3pt}{3pt}
%Tweak a bit the top margin
%\addtolength{\voffset}{-1.3cm}

%--------------------BEGIN DOCUMENT----------------------
\begin{document}

%--------------------TITLE-------------
\par{\centering
		{\Huge Hugo \textsc{Corbucci}
	}\bigskip\par}

%--------------------SECTIONS-----------------------------------
%Section: Personal Data
\section{Informações Pessoais}

\begin{tabular}{p{2.5cm}l}
  \textsc{Nacionalidade:} & Franco-Brasileiro
  \\
  \textsc{Nascimento:} & 26 de Dezembro de 1983\\
  \textsc{Mora em:}   & São Paulo, SP, Brasil \\
  \textsc{Email:}     &
  \href{mailto:hugo.corbucci@gmail.com}{hugo.corbucci@gmail.com}\\
  \textsc{Website:} & \href{http://hugocorbucci.com}{http://hugocorbucci.com}
\end{tabular}

% Section: Languages
\section{Línguas}
\begin{tabular}{p{2.5cm}l}
 \textsc{Português:}&Nativo\\
 \textsc{Francês:}&Nativo\\
 \textsc{Inglês:}&Fluente\\
 \textsc{Espanhol:}&Conhecimentos básicos\\
\end{tabular}

\section{Conhecimentos com Linguagens de Programação}
\begin{tabular}{p{2.5cm}l}
 Básico:& C, C++, Python, C\#, Haskell, Scala, Groovy\\
 Intermediário:& Java, Smalltalk, Clojure, Objective-C, Swift\\
 Avançado:& Go, Ruby, Javascript\\
\end{tabular}

\section{Interesses e Tópicos de Pesquisa}

\begin{itemize}
\item Desenvolvimento e evolução de produtos
\item Arquitetura de sistemas
\item Entrega Contínua
\item Experimentação Lean
\end{itemize}

\section{Destaques}

\begin{itemize}
\item Arquiteto de Software com ampla experiência em sistemas distribuídos, cloud/nuvem, compliance/regulamentações e planejamento arquitetural de longo prazo.
\item Programador generalista com interesse tanto nos aspectos
  técnicos quanto sociais da programação. Trabalhou em vários contextos desde aplicações \textit{web} até \textit{desktop}, linha de comando até interfaces
  gráficas passando por serviços.
\item Trabalha com equipes para obter o máximo de cada membro da equipe. Ajuda a desenvolver profissionais saídos direto da faculdade assim como outros com mais de 10 anos de experiência.
\item Líder técnico em equipes de experimentação lean e desenvolvimento de negócios.
\item Experiência em micro-serviços, automação de preparação de ambientes e arquiteturas para deploy contínuo.
\end{itemize}

\section{Outros interesses}

Triatlo, Escalada, Viajar, Vinhos, Cozinhar e Ficção Científica.

% Section: Work Experience
\section{Experiência de Trabalho}

\begin{tabular}{p{2.5cm}|p{13.5cm}}
  \emph{Mai 2023} & \textsc{DigitalOcean} - Engenheiro de \textit{Software} Principal\\
  \textsc{Set 2016}& \emph{Melhorando a colaboração e escala para nossos clientes}\\
  &\\
  &A DigitalOcean é um dos provedores de nuvem que cresceu e continua crescendo mais rápido no mercado. Ela provê produtos de base confiáveis, com alto desempenho e fácil uso para que nossos clientes possam tirar o melhor proveito da arquitetura na nuvem. Eu entrei na DigitalOcean quando ela estava próxima de 300 funcionários com um time de engenharia com aproximadamente 100 pessoas. Trabalhei remotamente por 3 meses antes de mudar para Nova Iorque. Eu fui parte da equipe antigamente responsável pela Experiência do Usuário e, em seguida, redirecionada para Contas \& Ferramentas de Produtividades. Se você já criou uma conta na DigitalOcean e usou qualquer serviço, você passou por parte do trabalho que meu time contribuiu para a plataforma da DigitalOcean. Mais recentemente, fui parte do time de Arquitetura com foco em direção de estratégica de longo prazo e iniciativas para todas as equipes de desenvolvimento da empresa com mais de 600 desenvolvedores. Fui o principal responsável em todos os trabalhos de controle de mudanças e adequação a regulamentações (como SOX) assim como o arquiteto responsável pelo programa de Identidade e Gestão de Acesso (IAM em inglês).
\end{tabular}


\begin{tabular}{p{2.5cm}|p{13.5cm}}
  \emph{Ago 2016} & \textsc{ThoughtWorks} - Consultor \textit{Lead}\\
  \textsc{Nov 2011}& \emph{Desenvolvedor}\\
  &\\
  &A ThoughtWorks é uma consultoria multinacional que ajuda organizações a entender e planejar como usar as inovações do mundo de TI e das tecnologias de software. Entrei na empresa quando ela já tinha 23 escritórios em 8 países no papel de consultor de desenvolvimento baseado em Chicago, IL. Isso me deu a oportunidade de trabalhar em projeto ágeis distribuídos com várias linguagens e tecnologias para vários tipos de plataformas. Desempenhei o papel de gerente de iteração para um time distribuído em 3 cidades e 3 fuso horários. Também atuei como líder técnico na mesma equipe com times de desenvolvimento entre 8 e 15 pessoas em cada cidade. Mais recentemente, após a mudança para São Francisco, tenho atuado como líder de desenvolvimento numa pequena equipe para experimentação lean responsável por validar ou invalidar ideias de negócio. A equipe mudou ao longo to tempo de um time de 4 pessoas para um grupo de 12 pessoas com 4 desenvolvedores. Também estive amplamente envolvido com o processo de recrutamento da ThoughtWorks com participações em uma grande quantitdade de revisões de código assim como entrevistas por telefone e face a face tanto para conhecimentos técnicos, de programação e de consultoria.
\end{tabular}

\begin{tabular}{p{2.5cm}|p{13.5cm}}
  \emph{Atual} & \textsc{Agile Alliance Brazil} - membro do comitê\\
  \textsc{Jan 2014}& \emph{Deliberar e executar a missão da alliança}\\
  &\\
  & Após 5 anos organizando a Agile Brazil, a Agile Alliance Brazil foi
  fundada para estabelecer fundações que permitam crescer a comunidade
  de desenvolvimento de software no Brazil e na América Latina. Além de
  garantir a continuidade da conferência Agile Brazil, a aliança espera
  apoiar outros programas na comunidade. Como membro fundador, faço parte
  do comitê diretor para estabelecer as práticas da aliança e prepará-la
  para renovação e sucesso.
\end{tabular}

\begin{tabular}{p{2.5cm}|p{13.5cm}}
  \emph{Atual} & \textsc{Agile Brazil} - organizador\\
  \textsc{Set 2009}& \emph{Webmaster, desenvolvedor, \textit{chair} de programa, membro do comitê avaliador e administrador de sistemas}\\
  &\\
  & Na Agile 2009, 8 Brasileiros se juntaram e decidiram organizar
  uma conferência nacional sobre métodos ágeis no Brasil. Atuei como
  membro do comitê organizador desde o início ajudando a formar o
  comitê de 15 pessoas, a planejar e implementar a estratégia de
  marketing, a entrar em contato com patrocinadores e a elaborar o
  programa da conferência. Em Junho de 2010, a conferência
  recebeu mais de 800 pessoas que submeteram mais de 150 propostas das
  quais em torno de 40 foram selecionadas. Trabalhei com o Danilo
  \textsc{Sato} no desenvolvimento do sistema de submissões em
  \textit{Ruby on Rails}. Desde então, o evento recebeu anualmente
  entre 700 e 1000 pessoas e tem crescido em qualidade e resultados
  para a comunidade.
\end{tabular}

\begin{tabular}{p{2.5cm}|p{13.5cm}}
  \textsc{Nov 2011} & \textsc{Agilbits} - fundador\\
  \textsc{Mar 2008}& \emph{Desenvolvedor e Consultor Ágil}\\
  &\\
  &A Agilbits foi criada para suprir as necessidades do mercado
  Brasileiro em termos de treinamento, \textit{coaching} e
  desenvolvimento de software de alta qualidade. Atuei como o
  administrador da empresa assim como desenvolvedor, consultor e
  \textit{coach}.
  Dentre os diversos projetos desenvolvidos, tivemos diversas
  experiências com \textit{Ruby on Rails} desde sistemas de gestão de
  conteúdo (\textit{CMS}) simples até processamento de imagens e
  sistemas de submissões.
  Também tivemos experiências em Java onde atuei como membro de uma
  equipe de 3 a 5 pessoas que liderou desde a fase de exploração de
  um projeto até o desenvolvimento completo de um produto em mais de
  25 iterações. O projeto é uma aplicação \textit{desktop} baseada na
  plataforma do Eclipse que atingiu mais de 65000 linhas de código de
  produção e 60000 linhas de testes automatizados.
\end{tabular}

\begin{tabular}{p{2.5cm}|p{13.5cm}}
  \textsc{Dez 2014} & \textsc{AgilCoop} - consultor\\
  \textsc{Jun 2007}& \emph{Instrutor e Coach}\\
  &\\
  &A AgilCoop foi fundada por professores, estudantes e ex-estudantes
  da Universidade do Estado de São Paulo (USP) para promover e
  divulgar os valores e princípios do desenvolvimento ágil de software
  no Brasil. Atuei como instrutor durante cursos de verão em cursos
  como ``Introdução a Métodos Ágeis de Desenvolvimento de Software'',
  ``Desenvolvimento de Software de Qualidade através de Testes
  Automatizados'' e ``Laboratório de Programação eXtrema'' em São
  Paulo, São Carlos e Salvador.
  Também trabalhei como \textit{coach} em diversos projetos na
  academia e na indústria. A AgilCoop também é responsável pela
  organização do ``Encontro Ágil'' do qual foi co-organizador em 2008,
  2009 e 2010. Esse evento tem como público alvo a comunidade local de
  pessoas interessadas em métodos ágeis que procuram por contato com
  pessoas mais experientes.
\end{tabular}

\begin{tabular}{p{2.5cm}|p{13.5cm}}
  \textsc{Jun 2010} & \textsc{Instituto de Matemática e Estatística
    (IME/USP)} - monitor\\
  \textsc{Mar 2007}& \emph{Monitor, \textit{Coach} de XP e programador}\\
  &\\
  & Fui monitor da disciplina de Programação Extrema (XP) por 3 anos
  seguidos. Atuei como \textit{Coach} e ajudei mais de 5 equipes em
  cada ano. Ajudei a implantar um scrum de scrums para permitir o
  aumento de inscritos na disciplina de 40 estudantes para mais de 60
  e de 5 equipes para mais de 8. Também fui monitor das disciplinas de
  Programação Orientada a Objetos e Programação Concorrente por 3 anos
  seguidos. Revisei e corrigi códigos de alunos de graduação e de
  pós-graduação assim como dei algumas aulas sob cada um dos assuntos.
\end{tabular}

\begin{tabular}{p{2.5cm}|p{13.5cm}}
  \textsc{Set 2007} & \textsc{MAPS Risk
    Management Solutions} - desenvolvedor\\
  \textsc{Dez 2006} &\emph{Desenvolvedor de Software and Coach Ágil}\\
  &\\
  & A MAPS é uma empresa que provê soluções para análise de risco do
  mercado financeiro. Eu me juntei à equipe que estava desenvolvendo o
  novo sistema destinado a substituir o antigo. Trabalhei numa equipe
  de 8 pessoas como desenvolvedor e \textit{coach} ágil para montar
  uma aplicação \textit{web} em Java usando \textit{Java Server
    Faces}. Ajudei a implantar um pouco de programação em par, testes
  de unidades e quadros de acompanhamento.
\end{tabular}

\begin{tabular}{p{2.5cm}|p{13.5cm}}
  \textsc{Dez 2006} & \textsc{Banco Votorantim} - estagiário\\
  \textsc{Dez 2005} &\emph{Desenvolvedor de Software}\\
  &\\
  &Entrei nessa equipe de 8 pessoas como o responsável de TI sob a
  supervisão de 2 analistas de negócio e 6 especialistas da área. Apesar de
  ter sido contratado para manter a aplicação existente, quando a
  maioria dos membros da equipe saiu da empresa, fiquei responsável
  por ajudar os analistas de negócio a manter a área funcionando.
\end{tabular}

\begin{tabular}{p{2.5cm}|p{13.5cm}}
  \textsc{Jun 2007} & \textsc{Instituto de Matemática e Estatística
    (IME/USP)} - \textit{coach} de XP\\
  \textsc{Mar 2006}& \emph{Líder de equipe e programador}\\
  &\\
  &  Trabalhei como \textit{Coach} de XP, líder de equipe e
  programador na disciplina de Programação Extrema (XP) por 2 anos
  seguidos. Ajudei duas equipes bem diferentes a desenvolver o
  \emph{Archimedes - O CAD Aberto}  do nada até uma aplicação baseada
  no SWT em Java e, então, evolui-la para uma aplicação baseada na
  plataforma do Eclipse. Coordenei uma equipe de 8 e depois de 7
  estudantes de graduação em 2006 e 2007 respectivamente.
\end{tabular}

\begin{tabular}{p{2.5cm}|p{13.5cm}}
  \textsc{Out 2005} & \textsc{Instituto de Matemática e Estatística
    (IME/USP)} - pesquisador de iniciação científica\\
  \textsc{Jan 2005}& \emph{Desenvolvedor de Ontologia}\\
  &\\
  & A Prof. Dra. Renata \textsc{Wasserman} e o Prof. Dr. Fabio
  \textsc{Kon} atuaram como meus orientadores para elaboração de uma
  ontologia sobre orientação a objetos. O trabalho serviu como base
  para uma busca semântica textual em alguns vídeos de aulas dadas por
  Joe Yoder.
\end{tabular}

% Section: Education
\section{Educação}
\begin{tabular}{p{2.5cm}l}
  \textsc{2011} & Mestrado em \textsc{Ciências da Computação}\\
  \textsc{2007} & \textbf{Universidade do Estado de São Paulo}, São Paulo\\
  & Dissertação: ``Métodos ágeis e software livre:\\
  & Um estudo do relacionamento entre estas duas comunidades''\\
  & \small Orientador: Prof. Alfredo \textsc{Goldman}\\
\end{tabular}

\begin{tabular}{p{2.5cm}l}
  \textsc{2006} & Graduação em \textsc{Ciências da Computação}\\
  \textsc{2003} &\normalsize\textbf{Universidade do Estado de São Paulo}, São Paulo\\
  & Dissertação: ``Archimedes: An Open Source CAD developed with\\
  & eXtreme Programming and Object Orientation''\\
  & \small Orientador: Fabio \textsc{Kon}\\
  & ``Aluno Destaque'' pela Sociedade Brasileira de Computação\\
\end{tabular}

\begin{tabular}{p{2.5cm}l}
  \textsc{2001} & \textbf{Lycée Pasteur}, São Paulo\\
  \textsc{1990} & Colégio Experimental bilíngue Português-Francês\\
  & Exame de aprovação Francês \textit{Baccalaureat Scientifique}: 14/20
\end{tabular}

\begin{tabular}{p{2.5cm}l}
  \textsc{1990} & \textbf{Escola Pública Primária Foyatier}, Paris\\
  \textsc{1986} & Escola primária no ensino Francês\\
\end{tabular}

\section{Publicações}

\begin{itemize}
\item \textbf{ThoughtWorks Antologia Brasil: Histórias de
    aprendizado e inovação - Capítulo sobre Padrões de Implementação em Ruby.} Hugo Corbucci.

  \textit{Casa do Código, São Paulo, Brasil.} Novembro 2014.

\item \textbf{Métodos Ágeis para Desenvolvimento de Software - Capítulo 2: "A História dos Métodos Ágeis no Brasil"} Hugo Corbucci,
        Alfredo Goldman, Fabio Kon, Claudia Melo e Viviane Santos.

  \textit{Bookman, São Paulo, Brasil.} Julho 2014.

\item \textbf{TDD em Ruby.} Hugo Corbucci e Maurício Aniche.

  \textit{Casa do Código, São Paulo, Brasil.} Abril 2014.

\item \textbf{Genesis and Evolution of the Agile Movement in Brazil
        – Perspective from Academia and Industry.} Hugo Corbucci,
        Alfredo Goldman, Eduardo Katayama, Fabio Kon, Claudia Melo
        e Viviane Santos.

  \textit{Anais do 25o Simpósio Brasileiro de Engenharia de Software
     (SBES 25), São Paulo, Brasil.} Julho 2011.

\item \textbf{Open Source and Agile Methods: Two worlds closer than it
    seems.} Hugo Corbucci, Alfredo Goldman.

  \textit{Anais da 11a Conferência Internacional sobre Desenvolvimento
    Ágil de Software (XP 2010), Trondheim, Noruega.} Junho 2010.

\item \textbf{Prototypes Are Forever: Evolving from a Prototype
    Project to a Full-Featured System} Hugo Corbucci, Mariana Vivian
  Bravo, Alexandre Freire da Silva, Fernando Freire da Silva.

  \textit{Anais da 11a Conferência Internacional sobre Desenvolvimento
    Ágil de Software (XP 2010), Trondheim, Noruega.} Junho 2010.

\item \textbf{Open Source and Agile: Two worlds that should have a
    closer interaction} Hugo Corbucci, Alfredo Goldman.

  \textit{Anais do V Workshop Latino-Americano sobre Engenharia de
    Software Experimental, 2008 (ESELAW 2008), Salvador, Brasil}
  Novembro 2008.

\item \textbf{Coding Dojo: an environment for learning and sharing
    Agile practices} Danilo Sato, Hugo Corbucci, Mariana Bravo.

  \textit{Anais da Conferência Ágil 2008 (Agile 2008), Toronto,
    Canadá.} Agosto 2008.

\item \textbf{Archimedes ­ o CAD Aberto: Uma aplicação para desenho
    técnico baseada na plataforma do Eclipse} Hugo Corbucci, Mariana
  Vivian Bravo.

  \textit{Anais do VII Workshop de Software Livre, 2007 (WSL 2007),
    Porto Alegre, Brasil.} Abril 2007.

\item \textbf{Archimedes: Um CAD Livre desenvolvido com programação
    extrema e orientação a objetos} Hugo Corbucci, Mariana Vivian
  Bravo. Orientador: Fabio Kon.

  \textit{Trabalho de conclusão de curso de Bacharelado em Ciências da
    Computação no IME/USP, São Paulo, Brasil.} Dezembro 2006.
\end{itemize}

\section{Atividades}

\begin{tabular}{p{2.5cm}l}
  Palestrante em: & \textbf{Agile Brazil 2015}, \textbf{Agile Brazil 2014}, \textbf{Agile Brazil 2013}, \textbf{Agile Brazil 2012},\\
  & \textbf{EclipseCon 2012}, \textbf{Ágiles 2011}, \textbf{Caipira Ágil 2011}, \textbf{Agile Brazil 2011},\\
  & \textbf{CONSEGI 2011}, \textbf{Encontro Ágil 2010}, \textbf{Agile Tour 2010 São Paulo}, \textbf{TDC 2010},\\
  & \textbf{DevInSampa 2010}, \textbf{Agile Brazil 2010}, \textbf{XP 2010}, \textbf{FISL 2009},\\
  & \textbf{1st Free Software Meeting Dataprev}, \textbf{4th SOLISC}, \textbf{Encontro Ágil 2009},\\
  & \textbf{2nd Locaweb Techday}, \textbf{InfoQ Brasil Launch Meeting}, \textbf{ESELAW 2008},\\
  & \textbf{PyCon Brasil 2008}, \textbf{Agile 2008}, \textbf{FISL 2008}, \textbf{Campus Party 2008}, \textbf{GrupySP}.
\end{tabular}

\begin{tabular}{p{2.5cm}l}
  Voluntário em: & \textbf{Agile Brazil 2012}, \textbf{Agile 2009}, \textbf{OOPSLA'08}, \textbf{Agile 2008}, \textbf{OOPSLA'07}.
\end{tabular}

\begin{tabular}{p{2.5cm}l}
  Revisor de Proposta: & \textbf{Agile Brazil 2015}, \textbf{Agile Brazil 2014}, \textbf{Agile Brazil 2013}, \textbf{Agile Brazil 2012},\\
  & \textbf{Agile Brazil 2011}, \textbf{Agile Brazil 2010}, \textbf{FISL 2009}, \textbf{FISL 2008}.
\end{tabular}

\begin{tabular}{p{2.5cm}l}
  Participante em: & \textbf{Agile 2015}, \textbf{Agile 2014}, \textbf{Agile 2013}, \textbf{Campus Party 2009}, \textbf{FISL 2007},\\
  & \textbf{FISL 2006}, \textbf{FISL 2005}.
\end{tabular}

\begin{tabular}{p{2.5cm}l}
  \textsc{Nov 2007} & 12o lugar na final Brasileira da
  \textbf{Maratona de Programação da ACM-ICPC}\\
  & com Jeferson \textsc{Rodrigues da Silva} e Marcio
  \textsc{Takashi Iura Oshiro}\\
  & em Belo Horizonte, MG, Brasil.\\
\end{tabular}

\begin{tabular}{p{2.5cm}l}
  \textsc{Jul 2007} & Co-fundador do \textbf{Coding Dojo São Paulo}\\
  & com Danilo \textsc{Sato} e Mariana \textsc{Bravo}.\\
\end{tabular}

\begin{tabular}{p{2.5cm}l}
  \textsc{Jan 2006} & Fundador do \textbf{Archimedes: o CAD Aberto}\\
  & usando Java e SWT.\\
\end{tabular}

\begin{tabular}{p{2.5cm}l}
  \textsc{Nov 2005} & 16o lugar na final Brasileira da
  \textbf{Maratona de Programação da ACM-ICPC}\\
  & com Jeferson \textsc{Rodrigues da Silva} e Marcio
  \textsc{Takashi Iura Oshiro}\\
  & em Riberão Preto, SP, Brasil.\\
\end{tabular}

\end{document}
