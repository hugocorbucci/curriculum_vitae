% !TEX program = XeLaTeX
\documentclass[letter,10pt]{article}

%A Few Useful Packages
\usepackage{marvosym}
\usepackage{fontspec} 					%for loading fonts
\usepackage{xunicode,xltxtra,url,parskip} 	%other packages for formatting
\RequirePackage{color,graphicx}
\usepackage[usenames,dvipsnames]{xcolor}
\usepackage{fullpage}
\usepackage{supertabular} 				%for Grades
\usepackage{titlesec} % custom \section
\usepackage{multicol}
\usepackage{multirow}

%Setup hyperref package, and colours for links
\usepackage{hyperref}
\definecolor{linkcolour}{rgb}{0,0.2,0.6}
\hypersetup{colorlinks,breaklinks,urlcolor=linkcolour, linkcolor=linkcolour}

%FONTS
\defaultfontfeatures{Mapping=tex-text}
%\setmainfont[SmallCapsFont = Fontin SmallCaps]{Fontin}

%CV Sections inspired by: 
%http://stefano.italians.nl/archives/26
\titleformat{\section}{\Large\scshape\raggedright}{}{0em}{}[\titlerule]
\titlespacing{\section}{0pt}{3pt}{3pt}
\pagestyle{empty}
%Tweak a bit the top margin
%\addtolength{\voffset}{-1.3cm}

%--------------------BEGIN DOCUMENT----------------------
\begin{document}

%--------------------TITLE-------------
\par{\centering
		{\Huge Hugo \textsc{Corbucci}
	}\bigskip\par}

%--------------------SECTIONS-----------------------------------
\begin{multicols}{2}
%Section: Personal Data
\section{Informações Pessoais}
\begin{tabular}{p{2.5cm}l}
  \textsc{Nacionalidade:} & Franco-Brasileiro\\
  \textsc{Nascimento:} & 26 de Dezembro de 1983\\
  \textsc{Mora em:}   & São Paulo, SP, Brasil \\
  \textsc{Email:}     &
  \href{mailto:hugo.corbucci@gmail.com}{hugo.corbucci@gmail.com}\\
  \textsc{Website:} & \href{http://hugocorbucci.com}{http://hugocorbucci.com}
\end{tabular}

% Section: Programming Languages
\section{Linguagens de Programação}
\begin{tabular}[t]{p{2.5cm}l}
  \multirow[t]{2}{*}{Básico:} & C, C++, Python,\\
 & Haskell, Scala, Groovy\\
  \multirow[t]{2}{*}{Intermediário:} & Java, Smalltalk, Clojure,\\
 & Objective-C, Swift\\
 Avançado:& Go, Ruby, Javascript\\
\end{tabular}
\end{multicols}

\begin{multicols}{2}
% Section: Interests
\section{Interesses}
\begin{itemize}
\item Desenvolvimento e evolução de produtos
\item Arquitetura de sistemas
\item Entrega Contínua
\item Experimentação Lean
\end{itemize}

% Section: Languages
\section{Línguas}
\begin{tabular}{p{2.5cm}l}
 \textsc{Português:}&Nativo\\
 \textsc{Francês:}&Nativo\\
 \textsc{Inglês:}&Fluente\\
 \textsc{Espanhol:}&Conhecimentos básicos\\
\end{tabular}
\end{multicols}


\section{Destaques}

\begin{itemize}
\item Arquiteto de Software com ampla experiência em sistemas distribuídos, \emph{cloud} (nuvem), \emph{compliance} (regulamentações) e planejamento arquitetural de longo prazo.
\item Programador generalista com interesse tanto nos aspectos
  técnicos quanto sociais da programação. Trabalhou em vários contextos desde aplicações \textit{web} até \textit{desktop}, linha de comando até interfaces
  gráficas passando por serviços e APIs.
\item Trabalha com equipes para obter o máximo de cada membro da equipe. Ajuda a desenvolver profissionais saídos direto da faculdade assim como outros com mais de 10 anos de experiência.
\item Líder técnico em equipes de experimentação lean e desenvolvimento de negócios.
\item Experiência em micro-serviços, automação de preparação de ambientes e arquiteturas para deploy contínuo.
\item Mestre em \textsc{Ciências da Computação} na \textbf{Universidade de São Paulo}, São Paulo, Brasil
\end{itemize}

% Section: Work Experience
\section{Experiência de Trabalho}

\textsc{DigitalOcean} - Engenheiro de Software \emph{Principal} - \textsc{09/2016}-\textsc{05/2023} - EUA e BR

Passei de engenheiro fullstack a líder técnico a, nos últimos 2.5 anos, arquiteto sendo responsável pelo \emph{design} geral dos sistemas da empresa que incluia 600 engenheiros. Também fui o principal contato para tudo que diz respeito a regulamentações e controle de mudanças relacionadas a SOX.

\textsc{ThoughtWorks} - Consultor \emph{Lead} - \textsc{11/2011}-\textsc{08/2016} - EUA, Canadá e Brasil

Trabalhei em vários projects pelos Estados Unidos, Canadá e Brasil com grandes empresas lidando tanto com sistemas legados assim como novos sistemas para averiguar valor de negócio.

\textsc{Agile Alliance Brazil \& Agile Brazil} - diretor e organizador - \textsc{09/2009}-\textsc{01/2021} - SP/SP

Como organizador fundador da conferência Agile Brazil, atuei em várias frentes para criar e crescer o que virou no seu auge a maior conferência de agilidade da América Latina. Participei tanto das equipes de programa quanto desenvolvendo sistemas usados pelas conferência e criando comitês para tocar a conferência.
Como diretor fundador da organização sem fins lucrativos Agile Alliance Brazil de 01/2014 até 11/2018, ajudei a desenhar os processos e atividades para manter a organização e crescer sua receita.

\end{document}
