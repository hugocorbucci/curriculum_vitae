\documentclass[letter,10pt]{article}

%A Few Useful Packages
\usepackage{marvosym}
\usepackage{fontspec} 					%for loading fonts
\usepackage{xunicode,xltxtra,url,parskip} 	%other packages for formatting
\RequirePackage{color,graphicx}
\usepackage[usenames,dvipsnames]{xcolor}
\usepackage{fullpage}
\usepackage{supertabular} 				%for Grades
\usepackage{titlesec} % custom \section

%Setup hyperref package, and colours for links
\usepackage{hyperref}
\definecolor{linkcolour}{rgb}{0,0.2,0.6}
\hypersetup{colorlinks,breaklinks,urlcolor=linkcolour, linkcolor=linkcolour}

%FONTS
\defaultfontfeatures{Mapping=tex-text}
%\setmainfont[SmallCapsFont = Fontin SmallCaps]{Fontin}

%CV Sections inspired by: 
%http://stefano.italians.nl/archives/26
\titleformat{\section}{\Large\scshape\raggedright}{}{0em}{}[\titlerule]
\titlespacing{\section}{0pt}{3pt}{3pt}
%Tweak a bit the top margin
%\addtolength{\voffset}{-1.3cm}

%--------------------BEGIN DOCUMENT----------------------
\begin{document}

%--------------------TITLE-------------
\par{\centering
		{\Huge Hugo \textsc{Corbucci}
	}\bigskip\par}

%--------------------SECTIONS-----------------------------------
%Section: Personal Data
      \section{Personal Data}

\begin{tabular}{p{2.5cm}l}
  \textsc{Nationality:} & French-Brazilian
  \\
  \textsc{Birth:} & December 26, 1983 \\
  \textsc{Living in:}   & Chicago, IL, USA \\
  \textsc{Email:}     &
  \href{mailto:hugo.corbucci@gmail.com}{hugo.corbucci@gmail.com}\\
  \textsc{Website:}     & \href{http://hugocorbucci.com}{http://hugocorbucci.com}
\end{tabular}

% Section: Languages
\section{Languages}
\begin{tabular}{p{2.5cm}l}
 \textsc{Portuguese:}&Native\\
 \textsc{French:}&Native\\
 \textsc{English:}&Fluent\\
 \textsc{Spanish:}&Basic Knowledge\\
 \textsc{Italian:}&Learning\\
\end{tabular}

\section{Programming Language Knowledge}
\begin{tabular}{p{2.5cm}l}
 Basic:& C, C++, Python, C\#, Haskell, Scala, Clojure, Groovy\\
 Intermediate:& Smalltalk, Objective-C, Swift\\
 Advanced:& Java, Ruby, Javascript\\
\end{tabular}

\section{Interests and Research Topics}

\begin{itemize}
\item Agile Methodologies
\item Open Source Software ``Eco-system''
\item DevOps
\item Automated testing frameworks/solutions
\item Distributed systems
\item Distributed teams
\item Dynamic Programming Languages
\item Continuous Delivery
\item Infrastructure as code and its testing solutions
\end{itemize}

\section{Highlights}

\begin{itemize}
\item Generalist programmer with interests in both the technical
  aspects and the social aspects of programming. Worked in many contexts from web to desktop, command-line to graphical to services.
\item Object Oriented programming knowledge as well as functional programming. Keywords are encapsulation, cohesion, design patterns, automated testing, mocking/stubbing, closures, immutable data, pattern matching, static type systems etc.
\item JVM knowledge: SWT, Eclipse Rich Client Platform, Maven, JUnit, Scala, Groovy, Graddle etc.
\item Web skills: Ruby on Rails, HTML5, CSS3, sass, scss, less, x-path, jquery, API development etc.
\item Experienced pair programmer and agile coach going through XP, Crystal, Scrum, Kanban, Lean etc.
\item Works with teams to maximize every team member's potential. Helps mentor professionals straight out of college as well as 6+ years of experience ones.
\end{itemize}

\section{Diversity}

Science Fiction Literature, Rock Climbing, Traveling, Wine, Cooking
Theatre and Cinema.

% Section: Work Experience
\section{Work Experience}

\begin{tabular}{p{2.5cm}|p{13.5cm}}
  \emph{Current} & \textsc{ThoughtWorks} - Lead Consultant\\
  \textsc{Nov 2011}& \emph{Developer}\\
  &\\
  &ThoughtWorks is a global consulting company that helps organizations understand and plan how to use the latest innovations in IT and software technology. I joined the company when it had 23 offices across 8 countries as a developer consultant based in Chicago, IL. It got me working on distributed agile projects with several languages and technologies for all sorts of platforms.
\end{tabular}

\begin{tabular}{p{2.5cm}|p{13.5cm}}
  \emph{Current} & \textsc{Agile Alliance Brazil} - board member\\
  \textsc{Jan 2014}& \emph{Deliberate as well as execute the alliance's mission}\\
  &\\
  & After 5 years organizing Agile Brazil, Agile Alliance Brazil
  was founded to provide the foundations to better grow the software
  development community in Brazil and Latin America. In addition to
  securing the continuity of the Agile Brazil conference, the alliance
  hopes to support other programs within the community. As a founding
  member, I am on the board to establish the alliance's practices
  and set it up for renewal and success.
\end{tabular}

\begin{tabular}{p{2.5cm}|p{13.5cm}}
  \emph{Current} & \textsc{Agile Brazil} - organizer\\
  \textsc{Sep 2009}& \emph{Webmaster, developer, program chair, review team
    member and system's administrator}\\
  &\\
  &At Agile 2009, about 8 Brazilians got together and
  decided to run a country-wide conference about Agile methods in
  Brazil. I acted as an organizing committee member from the start,
  helping to form the first 15 people committee, planning and implementing
  the marketing strategy, sponsoring contact and program of the conference.
  In June 2010, the conference received over 800 attendees which
  submitted over 150 proposals from which around 40 were selected. I
  worked with Danilo \textsc{Sato} to develop the submission system
  in Ruby on Rails. Since then, the event has steadily received between 700
  and 1000 people and has been improving in quality and outcomes for the
  community.
\end{tabular}

\begin{tabular}{p{2.5cm}|p{13.5cm}}
  \textsc{Nov 2011} & \textsc{Agilbits} - founder\\
  \textsc{Mar 2008}& \emph{Software Developer and Agile Consultant}\\
  &\\
  &Agilbits was created
  to supply needs of the Brazilian market in training, coaching as
  well as development of high quality software. I acted as the
  company administrator as well as software developer, consultant
  and coach.
  From the wide variety of projects developed, we had several
  experiences with Ruby on Rails from very simple content management
  systems up to image processing and submission systems.
  We also experiences with Java where I acted as member of a 3-5
  people team that led an exploration phase development up to a
  full-featured product running over 25 iterations. The project is a
  desktop application based on Eclipse's Rich Client Platform and
  reached over 40000 lines of production code and 45000 lines of
  automated tests.
\end{tabular}

\begin{tabular}{p{2.5cm}|p{13.5cm}}
  \textsc{Dez 2014} & \textsc{AgilCoop} consultant\\
  \textsc{Jun 2007}& \emph{Instructor and Coach}\\
  &\\
  &AgilCoop was founded by professors,
  students, and former students of the University of São Paulo (USP)
  to foster and spread the values and principles of agile software
  development in Brazil. I acted as an instructor during the summer
  courses on classes such as ``Introduction to Agile Methods'',
  ``Quality Software Development through Automated Tests'' and
  ``eXtreme Programming Lab'' in São Paulo, São Carlos and Salvador.
  I also worked as coach on several projects in the academia
  and in the industry. AgilCoop is also responsible for organizing
  the ``Agile Meeting'' (\emph{Encontro Ágil}), of which I was a
  co-organizer in 2008, 2009 and 2010. It is a local
  event aimed for people new to Agile methods looking for contacts
  with more experienced people.
\end{tabular}

\begin{tabular}{p{2.5cm}|p{13.5cm}}
  \textsc{Jun 2010} & \textsc{Instituto de Matemática e Estatística
    (IME/USP)} - Teaching Assistant\\
  \textsc{Mar 2007}& \emph{Teaching assistant, XP Coach and programmer}\\
  &\\
  & Teaching assistant on the eXtreme Programming (XP) discipline for
  3 years in a row. Coaching and helping over 5 teams on each
  year. Helped implanting a scrum of scrums to make it possible to
  increase from 40 students to over 60 and from 5 teams up to 8.
  Also 3 years being a Teching assitant for the Object Oriented
  Programming and Concurrent Programming disciplines for 3 years in a
  row. Reviewing and correcting codes from undergraduate and graduate
  students as well as giving some lessons on each subject.
\end{tabular}

\begin{tabular}{p{2.5cm}|p{13.5cm}}
  \textsc{Sep 2007} & \textsc{MAPS Risk
    Management Solutions} - Developer\\
  \textsc{Dec 2006} &\emph{Software Developer and Agile coach}\\
  &\\
  & MAPS is a software vendor providing market risk
  analysis solution. I joined the team that was building a new
  system to replace the legacy one. I worked on an 8 people team as a
  developer and agile coach to build a Java Web Application using
  Java Server Faces. I helped implanting some agile practices such as pair programming, unit
  testing and tracking.
\end{tabular}

\begin{tabular}{p{2.5cm}|p{13.5cm}}
  \textsc{Dec 2006} & \textsc{Banco Votorantim} - Intern\\
  \textsc{Dec 2005} &\emph{Software Developer}\\
  &\\
  &I joined this 8 people team as the IT responsible
  under 2 business analysts supervision and 6 business
  people. Although I was supposed to maintain a
  running application, when most members of the team left the company,
  I became responsible for helping the business analysts run the
  area.
\end{tabular}

\begin{tabular}{p{2.5cm}|p{13.5cm}}
  \textsc{Jun 2007} & \textsc{Instituto de Matemática e Estatística
    (IME/USP)} - XP Coach\\
  \textsc{Mar 2007}& \emph{Team leader and programmer}\\
  &\\
  &  Worked as an XP Coach, Team leader and programmer on the eXtreme
  Programming (XP) discipline for
  2 years in a row. I helped two very different teams develop
  \emph{Archimedes - The Open CAD} from scratch to a SWT based Java
  application and then evolve it to an Eclipse's Rich Client
  Application. Coordinated an 8 and 7 undergraduate students in
  2006 and 2007 respectively.
\end{tabular}

\begin{tabular}{p{2.5cm}|p{13.5cm}}
  \textsc{Oct 2005} & \textsc{Instituto de Matemática e Estatística
    (IME/USP)} - Undergraduate Researcher\\
  \textsc{Jan 2005}& \emph{Ontology developer}\\
  &\\
  & Prof. Renata \textsc{Wasserman} and Prof. Fabio \textsc{Kon} acted as my
  advisers to build an ontology about Object-Orientation. The work
  was to be used as a basis for a text based semantic search over a
  few classes given by Joe Yoder.
\end{tabular}

% Section: Education
\section{Education}
\begin{tabular}{p{2.5cm}l}
  \textsc{2011} & Master Student in \textsc{Computer Science}\\
  \textsc{2007} & \textbf{University of São Paulo}, São Paulo, Brazil\\
  & Thesis: ``Agile Practices in an Open Source Context''\\
  & \small Adviser: Prof. Alfredo \textsc{Goldman}\\
\end{tabular}

\begin{tabular}{p{2.5cm}l}
  \textsc{2006} & Undergraduate Degree in \textsc{Computer Science}\\
  \textsc{2003} &\normalsize\textbf{University of São Paulo}, São
  Paulo, Brazil\\
  & Thesis: ``Archimedes: An Open Source CAD developed with\\
  & eXtreme Programming and Object Orientation''\\
  & \small Adviser: Prof. Fabio \textsc{Kon}\\
  & ``Outstanding Student'' by the Brazilian Computer Society\\
\end{tabular}

\begin{tabular}{p{2.5cm}l}
  \textsc{2001} & \textbf{Lycée Pasteur}, São Paulo\\
  \textsc{1990} & Bilingual Portuguese-French College and High School\\
  & French High School exam \textit{Baccalaureat Scientifique}: 14/20
\end{tabular}

\begin{tabular}{p{2.5cm}l}
  \textsc{1990} & \textbf{Public Primary School Foyatier}, Paris\\
  \textsc{1986} & Primary school in France\\
\end{tabular}

\section{Publications}

\begin{itemize}
\item \textbf{ThoughtWorks Antologia Brasil: Histórias de
    aprendizado e inovação - Chapter about Ruby Implementation Patterns.} Hugo Corbucci.

  \textit{Casa do Código, São Paulo, Brasil.} November 2014.

\item \textbf{Métodos Ágeis para Desenvolvimento de Software - Chapter 2: "The history of Agile Methods in Brazil"} Hugo Corbucci,
        Alfredo Goldman, Fabio Kon, Claudia Melo and Viviane Santos.

  \textit{Bookman, São Paulo, Brasil.} July 2014.

\item \textbf{TDD em Ruby.} Hugo Corbucci and Maurício Aniche.

  \textit{Casa do Código, São Paulo, Brasil.} April 2014.

\item \textbf{Genesis and Evolution of the Agile Movement in Brazil
        – Perspective from Academia and Industry.} Hugo Corbucci,
        Alfredo Goldman, Eduardo Katayama, Fabio Kon, Claudia Melo
        and Viviane Santos.

  \textit{Proceedings for the 25th Brazilian Symposium of Software
     Engineering (SBES 25), São Paulo, Brasil.} July 2011.

\item \textbf{Open Source and Agile Methods: Two worlds closer than it
    seems.} Hugo Corbucci, Alfredo Goldman.

  \textit{Proceedings of the 11th International Conference on Agile
    Software Development (XP 2010), Trondheim, Norway.} June 2010.

\item \textbf{Prototypes Are Forever: Evolving from a Prototype
    Project to a Full-Featured System} Hugo Corbucci, Mariana Vivian
  Bravo, Alexandre Freire da Silva, Fernando Freire da Silva.

  \textit{Proceedings of the 11th International Conference on Agile
    Software Development (XP 2010), Trondheim, Norway.} June 2010.

\item \textbf{Open Source and Agile: Two worlds that should have a
    closer interaction} Hugo Corbucci, Alfredo Goldman.

  \textit{Proceedings of the V Experimental Software Engineering Latin
    American Workshop, 2008 (ESELAW 2008), Salvador, Brazil} November
  2008.

\item \textbf{Coding Dojo: an environment for learning and sharing
    Agile practices} Danilo Sato, Hugo Corbucci, Mariana Bravo.

  \textit{Proceedings of the Agile Conference 2008 (Agile 2008),
    Toronto, Canada.} August 2008.

\item \textbf{Archimedes ­ The Open CAD: A technical drawing
    application based on Eclipse's platform} Hugo Corbucci, Mariana
  Vivian Bravo.

  \textit{Proceedings of the VIII Free Software Workshop, 2007 (WSL
    2007), Porto Alegre, Brazil.} April 2007.

\item \textbf{Archimedes: An Open CAD developed with eXtreme
    Programming and Object-Orientation} Hugo Corbucci, Mariana Vivian
  Bravo. Adviser: Fabio Kon.

  \textit{Final Graduation Project, USP Institute of Mathematics and
    Statistics, São Paulo, Brazil.} December 2006.
\end{itemize}

\section{Activities}

\begin{tabular}{p{2.5cm}l}
  Speaker at: & \textbf{Agile Brazil 2014}, \textbf{Agile Brazil 2013}, \textbf{Agile Brazil 2012}, \textbf{EclipseCon 2012},\\
  & \textbf{Ágiles 2011}, \textbf{Caipira Ágil 2011}, \textbf{Agile Brazil 2011}, \textbf{CONSEGI 2011},\\
  & \textbf{Encontro Ágil 2010}, \textbf{Agile Tour 2010 São Paulo}, \textbf{TDC 2010}, \textbf{DevInSampa 2010},\\
  & \textbf{Agile Brazil 2010}, \textbf{XP 2010}, \textbf{1st Free Software Meeting Dataprev}, \textbf{4th SOLISC},\\
  & \textbf{Encontro Ágil 2009}, \textbf{FISL 2009}, \textbf{2nd Locaweb Techday},\\
  & \textbf{InfoQ Brasil Launch Meeting}, \textbf{ESELAW 2008}, \textbf{PyCon Brasil 2008}, \textbf{Agile 2008},\\
  & \textbf{FISL 2008}, \textbf{Campus Party 2008}, \textbf{GrupySP}.
\end{tabular}

\begin{tabular}{p{2.5cm}l}
  Volunteer at: & \textbf{Agile Brazil 2012}, \textbf{Agile 2009}, \textbf{OOPSLA'08}, \textbf{Agile 2008}, \textbf{OOPSLA'07}.
\end{tabular}

\begin{tabular}{p{2.5cm}l}
  Proposal reviewer: & \textbf{Agile Brazil 2014}, \textbf{Agile Brazil 2013}, \textbf{Agile Brazil 2012}, \textbf{Agile Brazil 2011},\\
  & \textbf{Agile Brazil 2010}, \textbf{FISL 2009}, \textbf{FISL 2008}.
\end{tabular}

\begin{tabular}{p{2.5cm}l}
  Attendee at: & \textbf{Agile 2014}, \textbf{Agile 2013}, \textbf{Campus Party 2009}, \textbf{FISL 2007},\\
  & \textbf{FISL 2006}, \textbf{FISL 2005}.
\end{tabular}

\begin{tabular}{p{2.5cm}l}
  \textsc{Nov 2007} & 12th place on the Brazilian finals of \textbf{ACM-ICPC
    Programming Marathon}\\
  & with Jeferson \textsc{Rodrigues da Silva} and Marcio
  \textsc{Takashi Iura Oshiro}\\
  & in Belo Horizonte, MG, Brazil.\\
\end{tabular}

\begin{tabular}{p{2.5cm}l}
  \textsc{Jul 2007} & Co-founded \textbf{Coding Dojo São Paulo}\\
  & with Danilo \textsc{Sato} and Mariana \textsc{Bravo}.\\
\end{tabular}

\begin{tabular}{p{2.5cm}l}
  \textsc{Jan 2006} & Founded \textbf{Archimedes: The Open CAD}\\
  & using Java and SWT.\\
\end{tabular}

\begin{tabular}{p{2.5cm}l}
  \textsc{Nov 2005} & 16th place on the Brazilian finals of \textbf{ACM-ICPC
    Programming Marathon}\\
  & with Jeferson \textsc{Rodrigues da Silva} and Marcio
  \textsc{Takashi Iura Oshiro}\\
  & in Riberão Preto, SP, Brazil.\\
\end{tabular}

\end{document}
