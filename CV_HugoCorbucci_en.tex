\documentclass[letter,10pt]{article}

%A Few Useful Packages
\usepackage{marvosym}
\usepackage{fontspec} 					%for loading fonts
\usepackage{xunicode,xltxtra,url,parskip} 	%other packages for formatting
\RequirePackage{color,graphicx}
\usepackage[usenames,dvipsnames]{xcolor}
\usepackage{fullpage}
\usepackage{supertabular} 				%for Grades
\usepackage{titlesec} % custom \section

%Setup hyperref package, and colours for links
\usepackage{hyperref}
\definecolor{linkcolour}{rgb}{0,0.2,0.6}
\hypersetup{colorlinks,breaklinks,urlcolor=linkcolour, linkcolor=linkcolour}

%FONTS
\defaultfontfeatures{Mapping=tex-text}
%\setmainfont[SmallCapsFont = Fontin SmallCaps]{Fontin}

%CV Sections inspired by: 
%http://stefano.italians.nl/archives/26
\titleformat{\section}{\Large\scshape\raggedright}{}{0em}{}[\titlerule]
\titlespacing{\section}{0pt}{3pt}{3pt}
%Tweak a bit the top margin
%\addtolength{\voffset}{-1.3cm}

%--------------------BEGIN DOCUMENT----------------------
\begin{document}

%--------------------TITLE-------------
\par{\centering
		{\Huge Hugo \textsc{Corbucci}
	}\bigskip\par}

%--------------------SECTIONS-----------------------------------
%Section: Personal Data
      \section{Personal Data}

\begin{tabular}{p{2.5cm}l}
  \textsc{Nationality:} & French-Brazilian
  \\
  \textsc{Birth:} & December 26, 1983 \\
  \textsc{Living in:}   & São Paulo, Brazil \\
  \textsc{Email:}     &
  \href{mailto:hugo.corbucci@gmail.com}{hugo.corbucci@gmail.com}\\
  \textsc{Blog:}     & \href{http://codeache.blogspot.com}{http://codeache.blogspot.com}
\end{tabular}

% Section: Education
\section{Education}
\begin{tabular}{p{2.5cm}l}
  \emph{Current} & Master Student in \textsc{Computer Science}\\
  \textsc{2007} & \textbf{University of São Paulo}, São Paulo, Brazil\\
  & Thesis: ``Agile Practices in an Open Source Context''\\
  & \small Adviser: Prof. Alfredo \textsc{Goldman}\\
\end{tabular}

\begin{tabular}{p{2.5cm}l}
  \textsc{2006} & Undergraduate Degree in \textsc{Computer Science}\\
  \textsc{2003} &\normalsize\textbf{University of São Paulo}, São
  Paulo, Brazil\\
  & Thesis: ``Archimedes: An Open Source CAD developed with\\
  & eXtreme Programming and Object Orientation''\\
  & \small Adviser: Fabio \textsc{Kon}\\
  & ``Outstanding Student'' by the Brazilian Computer
  Science Society\\
\end{tabular}

\begin{tabular}{p{2.5cm}l}
  \textsc{2001} & \textbf{Lycée Pasteur}, São Paulo\\
  \textsc{1990} & Bilingual Portuguese-French College and High School\\
  & French High School exam \textit{Baccalaureat Scientifique}: 14/20
\end{tabular}

\begin{tabular}{p{2.5cm}l}
  \textsc{1990} & \textbf{Public Primary School Foyatier}, Paris\\
  \textsc{1986} & Primary school in France\\
\end{tabular}

% Section: Languages
\section{Languages}
\begin{tabular}{p{2.5cm}l}
 \textsc{Portuguese:}&Native\\
 \textsc{French:}&Native\\
 \textsc{English:}&Fluent\\
 \textsc{Spanish:}&Basic Knowledge\\
\end{tabular}

\section{Programming Language Knowledge}
\begin{tabular}{p{2.5cm}l}
 Basic:& C, C++, Python, C\#, Haskell\\
 Intermediate:& Smalltalk, Javascript\\
 Advanced:& Java, Ruby\\
\end{tabular}

\section{Interests and Research Topics}

\begin{itemize}
\item Agile Methodologies
\item Open Source Software ``Eco-system''
\item Coding Dojo
\item Integrated Development Environments
\item Clean Code
\item Graphical User Automated Tests
\item Multi-language Applications
\item Fail Recovering Applications
\item Programming Languages with syntax closer to natural language ones
\end{itemize}

\section{Highlights}

\begin{itemize}
\item Generalist programmer with interests in both the technical
  aspects and the social aspects of programming.
\item Software developer in a range of applications (from web to
  desktop, command-line to graphical to services).
\item Highly skilled in Object Oriented programming (encapsulation,
  cohesion, design patterns, UML, unit testing, integration testing,
  mocking/stubbing and conventions).
\item Highly skilled in Java technologies (SWT, Eclipse Rich Client
  Platform, Ant, JUnit, Swing, VRaptor).
\item Highly skilled in Web technologies (Ruby on Rails, VRaptor, GWT,
  AJAX, JSON, Javascript, CSS, HTML, haml, sass, x-path, jquery,
  prototype).
\item Experienced pair programmer and agile coach.
\item Used to formal and very informal work environments.
\end{itemize}

% Section: Work Experience
\section{Work Experience}

\begin{tabular}{p{2.5cm}|p{13.5cm}}
  \emph{Current} & \textsc{Agilbits} founder\\
  \textsc{Mar 2008}& \emph{Software Developer and Agile Consultant}\\
  &\\
  &Agilbits was created
  to supply needs of the Brazilian market in training, coaching as
  well as development of high quality software. I acted as the
  company administrator as well as software developer, consultant
  and coach.
  From the wide variety of projects developed, we had several
  experiences with Ruby on Rails for very simple content management
  systems up to image processing and submission systems.
  On the Java side, I acted as member of a 3-5 people team that led an
  exploration phase development up to a full-featured product running
  over 25 iterations. The project is a desktop application based on Eclipse's Rich Client
  Platform and reached over 40000 lines of production code and 45000
  lines of automated tests.
\end{tabular}

\begin{tabular}{p{2.5cm}|p{13.5cm}}
  \emph{Current} & \textsc{AgilCoop} consultant\\
  \textsc{Jan 2007}& \emph{Instructor and Coach}\\
  &\\
  &AgilCoop was founded by professors,
  students, and former students of the University of São Paulo (USP)
  to foster and spread the values and principles of agile software
  development in Brazil. I acted as an instructor during the summer
  courses on classes such as ``Introduction to Agile Methods'',
  ``Quality Software Development through Automated Tests'' and
  ``eXtreme Programming Lab'' in São Paulo, São Carlos and Salvador.
  I also worked as coach on several projects in the academia
  and in the industry. AgilCoop is also responsible for organizing
  the ``Agile Meeting'' (\emph{Encontro Ágil}) from which I was a
  co-organizer in 2008, 2009 and 2010. This event is a local
  event aimed to people new to Agile methods looking for contacts
  with more experienced people.
\end{tabular}

\begin{tabular}{p{2.5cm}|p{13.5cm}}
  \emph{Current} & \textsc{Agile Brazil} organizer\\
  \textsc{Sep 2009}& \emph{Webmaster, developer and review team
    member}\\
  &\\
  &At Agile 2009, about 8 Brazilians got together and
  decided to run a country wide conference about Agile methods in
  Brazil. I acted as an organizing committee member from start helping to form the
  15 people committee, planning and implementing the marketing
  strategy, sponsoring contact and program of the conference. In
  June 2010, the conference received over 800 attendees which
  submitted over 150 proposals from which around 40 were selected. I
  worked with Danilo \textsc{Sato} to develop the submission system in Ruby on
  Rails.
\end{tabular}

\begin{tabular}{p{2.5cm}|p{13.5cm}}
  \textsc{Mar - Jun} & \textsc{Instituto de Matemática e Estatística
    (IME/USP)} - Teaching Assistant\\
  \textsc{2008 to 2010}& \emph{XP Coach and programmer}\\
  &\\
  & Teaching assistant on the eXtreme Programming (XP) discipline for
  3 years in a row. Coaching and helping over 5 teams on each
  year. Helped implanting a scrum of scrum to make it possible to
  increase from 40 students to over 60 and from 5 teams up to 8.
\end{tabular}

\begin{tabular}{p{2.5cm}|p{13.5cm}}
  \textsc{Mar - Jun} & \textsc{Instituto de Matemática e Estatística
    (IME/USP)} - Teaching Assistant\\
  \textsc{2007 to 2009}& \emph{Speaker, code reviewer and code critique}\\
  &\\
  & Teaching assistant on the Object Oriented Programming and
  Concurrent Programming disciplines for
  3 years in a row. Reviewing and correcting codes from undergraduate
  and graduate students as well as giving some talk on each subject.
\end{tabular}

\begin{tabular}{p{2.5cm}|p{13.5cm}}
  \textsc{Sep 2007} & Developer at \textsc{MAPS Risk
    Management Solutions} \\
  \textsc{Dec 2006} &\emph{Software Developer and Agile coach}\\
  &\\
  & MAPS is a software vendor providing market risk
  analysis solution. I joined the team that was building a new
  system to replace the legacy one. I worked on a 8 people team as a
  developer and agile coach to build a Java Web Application using
  Java Server Faces. Helped implanting some pair programming, unit
  testing and tracking.
\end{tabular}

\begin{tabular}{p{2.5cm}|p{13.5cm}}
  \textsc{Dec 2006} & \textsc{Banco Votorantim} - Intern\\
  \textsc{Dec 2005} &\emph{Software Developer}\\
  &\\
  &I joined this 8 people team as the IT responsible
  under 2 business analysts supervision and 6 business
  people. Although I was supposed to maintain a
  running application, when most members of the team left the company,
  I became responsible for helping the business analysts run the
  area.
\end{tabular}

\begin{tabular}{p{2.5cm}|p{13.5cm}}
  \textsc{Mar - Jun} & \textsc{Instituto de Matemática e Estatística
    (IME/USP)} - XP Coach\\
  \textsc{2006 and 2007}& \emph{Team leader and programmer}\\
  &\\
  &  Worked as an XP Coach, Team leader and programmer on the eXtreme
  Programming (XP) discipline for
  2 years in a row. I helped two very different teams develop
  \emph{Archimedes - The Open CAD} from scratch to a SWT based Java
  application and then evolve it to an Eclipse's Rich Client
  Application. Coordinated an 8 and 7 undergraduate students in
  2006 and 2007 respectively.
\end{tabular}

\begin{tabular}{p{2.5cm}|p{13.5cm}}
  \textsc{Oct 2005} & \textsc{Instituto de Matemática e Estatística
    (IME/USP)} - Undergraduate Researcher\\
  \textsc{Jan 2005}& \emph{Ontology developer}\\
  &\\
  & Prof. Dr. Renata \textsc{Wasserman} and Prof. Dr. Fabio \textsc{Kon} acted as my
  advisers to build an ontology about Object-Orientation. The work
  was to be used as a basis for a text based semantic search over a
  few classes given by Joe Yoder.
\end{tabular}

\begin{tabular}{p{2.5cm}|p{13.5cm}}
  \textsc{Dec 2003} & \textsc{City of Knowledge} - Internship\\
  \textsc{Jun 2003}& \emph{Webmaster}\\
  &\\
  & I worked as a webmaster for the City of Knowledge (\textit{Cidade
    do Conhecimento}) mantaining an elearning system developed in PHP.
\end{tabular}

\section{Publications}

\begin{itemize}
\item \textbf{Open Source and Agile Methods: Two worlds closer than it
    seems.} Hugo Corbucci, Alfredo Goldman.

  \textit{Proceedings of the 11th International Conference on Agile
    Software Development (XP 2010), Trondheim, Norway.} June 2010.

\item \textbf{Prototypes Are Forever: Evolving from a Prototype
    Project to a Full-Featured System} Hugo Corbucci, Mariana Vivian
  Bravo, Alexandre Freire da Silva, Fernando Freire da Silva.

  \textit{Proceedings of the 11th International Conference on Agile
    Software Development (XP 2010), Trondheim, Norway.} June 2010.

\item \textbf{Open Source and Agile: Two worlds that should have a
    closer interaction} Hugo Corbucci, Alfredo Goldman.

  \textit{Proceedings of the V Experimental Software Engineering Latin
    American Workshop, 2008 (ESELAW 2008), Salvador, Brazil} November
  2008.

\item \textbf{Coding Dojo: an environment for learning and sharing
    Agile practices} Danilo Sato, Hugo Corbucci, Mariana Bravo.

  \textit{Proceedings of the Agile Conference 2008 (Agile 2008),
    Toronto, Canada.} August 2008.

\item \textbf{Archimedes ­ The Open CAD: A technical drawing
    application based on Eclipse's platform} Hugo Corbucci, Mariana
  Vivian Bravo.

  \textit{Proceedings of the VIII Free Software Workshop, 2007 (WSL
    2007), Porto Alegre, Brazil.} April 2007.

\item \textbf{Archimedes: An Open CAD developed with eXtreme
    Programming and Object-Orientation} Hugo Corbucci, Mariana Vivian
  Bravo. Adviser: Fabio Kon.

  \textit{Final Graduation Project, USP Institute of Mathematics and
    Statistics, São Paulo, Brazil.} December 2006.
\end{itemize}

\section{Activities}

\begin{tabular}{p{2.5cm}l}
  \textsc{Aug 2010} & Speaker at \textbf{TDC 2010} Conference\\
  &Giving a talk about ``Clean Code'' in São Paulo, SP, Brazil.\\
\end{tabular}

\begin{tabular}{p{2.5cm}l}
  \textsc{Aug 2010} & Speaker at \textbf{DevInSampa 2010} Conference\\
  &Giving a talk about ``Clean Code'' in São Paulo, SP, Brazil.\\
\end{tabular}

\begin{tabular}{p{2.5cm}l}
  \textsc{Jun 2010} & Speaker at \textbf{Agile Brazil 2010} Conference\\
  &Giving a tutorial about ``Agile retrospectives'' in Porto Alegre, RS, Brazil.\\
\end{tabular}

\begin{tabular}{p{2.5cm}l}
  \textsc{Jun 2010} & Speaker at \textbf{XP 2010}\\
  &Giving a 180 minutes workshop for ``Coding Dojo - Kake Format'',\\
  & presenting an experience report ``Prototypes are forever''\\
  & with Alexandre \textsc{Freire} and Mariana \textsc{Bravo},\\
  & and a Kata with Danilo \textsc{Sato} in Smalltalk in Trondheim, Norway.\\
\end{tabular}

\begin{tabular}{p{2.5cm}l}
  \textsc{Dec 2009} & Speaker at the \textbf{1st Free Software Meeting Dataprev}\\
  &Giving a talk about ``Agile Methods in Free software contexts''\\
  &in Campo Grande, MS, Brazil.\\
\end{tabular}

\begin{tabular}{p{2.5cm}l}
  \textsc{Dec 2009} & Speaker at the \textbf{4th SOLISC}\\
  &Giving a talk about ``Archimedes - The Open CAD'' in Florianópolis, SC, Brazil.\\
\end{tabular}

\begin{tabular}{p{2.5cm}l}
  \textsc{Nov 2009} & Speaker at \textbf{ESELAW 2008}\\
  &Giving a course about ``Agile Methodologies'' and\\
  &a talk about ``Agile Methods in Free software contexts'' in Salvador, BA, Brazil.\\
\end{tabular}

\begin{tabular}{p{2.5cm}l}
  \textsc{Oct 2009} & Speaker at \textbf{Encontro Ágil 2009}\\
  &Giving a talk about ``Agile Methods in Free software contexts'' and\\
  &a workshop about ``Lean Lego Game'' from the work of\\
  &Francisco \textsc{Trindade} and Danilo \textsc{Sato} in São Paulo, SP, Brazil.\\
\end{tabular}

\begin{tabular}{p{2.5cm}l}
  \textsc{Aug 2009} & Volunteer at \textbf{Agile 2009}\\
  & with the Academic Grants in Chicago, IL, USA.\\
\end{tabular}

\begin{tabular}{p{2.5cm}l}
  \textsc{Jun 2009} & Speaker at \textbf{FISL 2009}\\
  &Giving a talk about ``Archimedes - The Open CAD'',\\
  & a tutorial about ``Eclipse RCP - Native desktop applications'' and\\
  & a workshop about ``Coding Dojo''\\
  & at the 10th International Forum for Free Software in Porto
  Alegre, RS, Brazil.
\end{tabular}

\begin{tabular}{p{2.5cm}l}
  \textsc{Apr 2009} & Proposals reviewer for \textbf{FISL 2009}.\\
\end{tabular}

\begin{tabular}{p{2.5cm}l}
  \textsc{Fev 2009} & Speaker at \textbf{2nd Locaweb Techday}\\
  & Giving a Code Kata about ``Rails + RSpec + Cucumber'' in São Paulo, SP, Brazil.\\
\end{tabular}

\begin{tabular}{p{2.5cm}l}
  \textsc{Jan 2009} & Attended \textbf{Campus Party 2009}\\
  &in São Paulo, SP, Brazil.\\
\end{tabular}

\begin{tabular}{p{2.5cm}l}
  \textsc{Oct 2008} & Student Volunteer at \textbf{OOPSLA'08}\\
  & in the Gold program in Nashville, TN, USA.\\
\end{tabular}

\begin{tabular}{p{2.5cm}l}
  \textsc{Sep 2008} & Speaker at \textbf{PyCon Brasil 2008}\\
  &Giving a workshop about ``Coding Dojo'' in Rio de Janeiro, RJ, Brazil.\\
\end{tabular}

\begin{tabular}{p{2.5cm}l}
  \textsc{Aug 2008} & Speaker and Volunteer at \textbf{Agile 2008}\\
  &Presenting an experience report ``Coding Dojo: An Environment for
  Learning and\\
  & Sharing Agile Practices'' with Danilo \textsc{Sato} and Mariana \textsc{Bravo} and\\
  & working as a volunteer with the Academic Grants in Toronto, ON, Canada.\\
\end{tabular}

\begin{tabular}{p{2.5cm}l}
  \textsc{Jun 2008} & Speaker at \textbf{FISL 2008}\\
  &Giving a workshop about ``Coding Dojo''\\
  & at the 9th International Forum for Free Software in Porto
  Alegre, RS, Brazil.\\
\end{tabular}

\begin{tabular}{p{2.5cm}l}
  \textsc{Apr 2008} & Proposals reviewer for \textbf{FISL 2008}.\\
\end{tabular}

\begin{tabular}{p{2.5cm}l}
  \textsc{Mar 2008} & Speaker at \textbf{Campus Party 2008}\\
  &Giving a workshop about ``Archimedes - The Open CAD'' in São Paulo, SP, Brazil.\\
\end{tabular}

\begin{tabular}{p{2.5cm}l}
  \textsc{Jan 2008} & Speaker at \textbf{GrupySP}\\
  &Giving a workshop about ``Coding Dojo'' in São Paulo, SP, Brazil.\\
\end{tabular}

\begin{tabular}{p{2.5cm}l}
  \textsc{Nov 2007} & 12th place on the Brazilian finals of \textbf{ACM-ICPC
    Programming Marathon}\\
  & with Jeferson \textsc{Rodrigues da Silva} and Marcio
  \textsc{Takashi Iura Oshiro}\\
  & in Belo Horizonte, MG, Brazil.\\
\end{tabular}

\begin{tabular}{p{2.5cm}l}
  \textsc{Oct 2007} & Student Volunteer at \textbf{OOPSLA'07}\\
  & in the Gold program in Montreal, QC, Canada.\\
\end{tabular}

\begin{tabular}{p{2.5cm}l}
  \textsc{May 2007} & Attended at \textbf{FISL 2007}\\
  & the 8th International Forum for Free Software in Porto
  Alegre, RS, Brazil.\\
\end{tabular}

\begin{tabular}{p{2.5cm}l}
  \textsc{Jul 2007} & Co-founded \textbf{Coding Dojo São Paulo}\\
  & with Danilo \textsc{Sato} and Mariana \textsc{Bravo}.\\
\end{tabular}

\begin{tabular}{p{2.5cm}l}
  \textsc{Apr 2006} & Attended \textbf{FISL 2006}\\
  & the 7th International Forum for Free Software in Porto
  Alegre, RS, Brazil.\\
\end{tabular}

\begin{tabular}{p{2.5cm}l}
  \textsc{Jan 2006} & Founded \textbf{Archimedes: The Open CAD}\\
  & using Java and SWT.\\
\end{tabular}

\begin{tabular}{p{2.5cm}l}
  \textsc{Nov 2005} & 16th place on the Brazilian finals of \textbf{ACM-ICPC
    Programming Marathon}\\
  & with Jeferson \textsc{Rodrigues da Silva} and Marcio
  \textsc{Takashi Iura Oshiro}\\
  & in Riberão Preto, SP, Brazil.\\
\end{tabular}

\begin{tabular}{p{2.5cm}l}
  \textsc{Mar 2005} & Attended \textbf{FISL 2005}\\
  & the 6th International Forum for Free Software in Porto
  Alegre, RS, Brazil.\\
\end{tabular}

\section{Diversity}

Science Fiction Literature, Rock Climbing, Traveling,
Theatre, Cinema.

\end{document}
